%% This template can be used to write a paper for
%% Computer Physics Communications using LaTeX.
%% For authors who want to write a computer program description,
%% an example Program Summary is included that only has to be
%% completed and which will give the correct layout in the
%% preprint and the journal.
%% The `elsarticle' style is used and more information on this style
%% can be found at 
%% http://www.elsevier.com/wps/find/authorsview.authors/elsarticle.
%%
%%
\documentclass[final,1p,twocolumn]{elsarticle}

%% Use the option review to obtain double line spacing
%% \documentclass[preprint,review,12pt]{elsarticle}

%% Use the options 1p,twocolumn; 3p; 3p,twocolumn; 5p; or 5p,twocolumn
%% for a journal layout:
%% \documentclass[final,1p,times]{elsarticle}
%% \documentclass[final,1p,times,twocolumn]{elsarticle}
%% \documentclass[final,3p,times]{elsarticle}
%% \documentclass[final,3p,times,twocolumn]{elsarticle}
%% \documentclass[final,5p,times]{elsarticle}
%% \documentclass[final,5p,times,twocolumn]{elsarticle}

%% if you use PostScript figures in your article
%% use the graphics package for simple commands
%% \usepackage{graphics}
%% or use the graphicx package for more complicated commands
%% \usepackage{graphicx}
%% or use the epsfig package if you prefer to use the old commands
%% \usepackage{epsfig}

%% The amssymb package provides various useful mathematical symbols
\usepackage{amssymb}
\usepackage[usenames]{color}
%% The amsthm package provides extended theorem environments
%% \usepackage{amsthm}

%% The lineno packages adds line numbers. Start line numbering with
%% \begin{linenumbers}, end it with \end{linenumbers}. Or switch it on
%% for the whole article with \linenumbers after \end{frontmatter}.
%% \usepackage{lineno}

%% natbib.sty is loaded by default. However, natbib options can be
%% provided with \biboptions{...} command. Following options are
%% valid:

%%   round  -  round parentheses are used (default)
%%   square -  square brackets are used   [option]
%%   curly  -  curly braces are used      {option}
%%   angle  -  angle brackets are used    <option>
%%   semicolon  -  multiple citations separated by semi-colon
%%   colon  - same as semicolon, an earlier confusion
%%   comma  -  separated by comma
%%   numbers-  selects numerical citations
%%   super  -  numerical citations as superscripts
%%   sort   -  sorts multiple citations according to order in ref. list
%%   sort&compress   -  like sort, but also compresses numerical citations
%%   compress - compresses without sorting
%%
%% \biboptions{comma,round}

% \biboptions{}

%% This list environment is used for the references in the
%% Program Summary
%%
\newcounter{bla}
\newenvironment{refnummer}{%
\list{[\arabic{bla}]}%
{\usecounter{bla}%
 \setlength{\itemindent}{0pt}%
 \setlength{\topsep}{0pt}%
 \setlength{\itemsep}{0pt}%
 \setlength{\labelsep}{2pt}%
 \setlength{\listparindent}{0pt}%
 \settowidth{\labelwidth}{[9]}%
 \setlength{\leftmargin}{\labelwidth}%
 \addtolength{\leftmargin}{\labelsep}%
 \setlength{\rightmargin}{0pt}}}
 {\endlist}

\journal{Computer Physics Communications}

\begin{document}

\begin{frontmatter}

%% Title, authors and addresses

%% use the tnoteref command within \title for footnotes;
%% use the tnotetext command for the associated footnote;
%% use the fnref command within \author or \address for footnotes;
%% use the fntext command for the associated footnote;
%% use the corref command within \author for corresponding author footnotes;
%% use the cortext command for the associated footnote;
%% use the ead command for the email address,
%% and the form \ead[url] for the home page:
%%
%% \title{Title\tnoteref{label1}}
%% \tnotetext[label1]{}
%% \author{Name\corref{cor1}\fnref{label2}}
%% \ead{email address}
%% \ead[url]{home page}
%% \fntext[label2]{}
%% \cortext[cor1]{}
%% \address{Address\fnref{label3}}
%% \fntext[label3]{}

\title{The TDHF Code Sky3D Version 1.1} 

\author{B.~Schuetrumpf\corref{gsi} }\ead{b.schuetrumpf@gsi.de}
\address{National Superconducting Cyclotron Laboratory\\
Michigan State University, East Lansing, Michigan 48824, USA}

\author{P.-G.~Reinhard}\ead{Paul-Gerhard.Reinhard@physik.uni-erlangen.de}
\address{Institut f\"ur Theoretische Physik II, Universit\"at
  Erlangen-N\"urnberg, \\Staudtstrasse 7, 91058 Erlangen, Germany}

\author{P.~D.~Stevenson}\ead{p.stevenson@surrey.ac.uk}
\address{Department of Physics, University of Surrey, Guildford,
  Surrey, GU2 7XH, United Kingdom}

\author{A. S. Umar}\ead{sait.a.umar@Vanderbilt.Edu}
\address{Department of Physics and Astronomy, Vanderbilt University, \\Nashville, Tennessee 37235, USA}

\author{J.~A.~Maruhn\corref{author}} \ead{maruhn@th.physik.uni-frankfurt.de}
\address{Institut f\"ur Theoretische Physik, Goethe-Universit\"at,
  Max-von-Laue-Str. 1, \\60438 Frankfurt am Main, Germany}

\cortext[gsi]{Present address: Gesellschaft f\"ur
    Schwerionenforschung, 64291 Darmstadt, Germany}
\cortext[author]{Corresponding author.}

\begin{abstract}
  The nuclear mean-field model based on Skyrme forces or related
  density functionals has found widespread application to the
  description of nuclear ground states, collective vibrational
  excitations, and heavy-ion collisions. The code Sky3D solves the
  static or dynamic equations on a three-dimensional Cartesian mesh
  with isolated or periodic boundary conditions and no further
  symmetry assumptions. Pairing can be included in the BCS
  approximation for the static case. The code is implemented with a
  view to allow easy modifications for including additional physics or
  special analysis of the results.
\end{abstract}

\begin{keyword}
Hartree-Fock; BCS; Density-functional theory; Skyrme energy functional; Giant Resonances;
Heavy-Ion collisions.
\end{keyword}
\end{frontmatter}

{\bf NEW VERSION PROGRAM SUMMARY}
  %Delete as appropriate.
{\bf PROGRAM SUMMARY}

\begin{small}
\noindent
{\em Title:}  The TDHF Code Sky3D Version 1.1\\
{\em Authors:}   B.~Schuetrumpf, P.-G.~Reinhard, P.~D.~Stevenson,
A.~S.~Umar, and J.~A.~Maruhn, \\
{\em Program Title:} Sky3D\\
{\em Journal Reference:}                                      \\
  %Leave blank, supplied by Elsevier.
{\em Catalogue identifier:}                                   \\
  %Leave blank, supplied by Elsevier.
{\em Licensing provisions:}   none\\
  %enter "none" if CPC non-profit use license is sufficient.
{\em Programming language:} Fortran 90. The {\tt OpenMP} version
requires a relatively recent compiler; it was found to work using {\tt
  gfortran 4.6.2} or later and the Intel compiler version 12 or later.\\
{\em Computer:}   All computers with a Fortran compiler supporting at
least Fortran 90.\\
  %Computer(s) for which program has been designed.
{\em Operating system:}  All operating systems with such a
compiler. Some of the Makefiles and scripts depend on a Unix-like
system and need modification under Windows.\\
  %Operating system(s) for which program has been designed.
{\em RAM:} 1~GB\\
  %RAM in bytes required to execute program with typical data.
{\em Number of processors used:} no built-in limit, parallelization
using both OpenMP and MPI.\\
  %If more than one processor.
{\em Supplementary material:} Extensive documentation and a number of
utility programs to analyze the results and prepare them for graphics
output using the Silo library ({\tt
  http://wci.llnl.gov/simulation/computer-codes/silo}) for use in VisIT
\cite{HPV:VisIt} or Paraview ({\tt https://www.paraview.org}). The
code can serve as a template for interfacing to other database or graphics systems.  \\
{\em Keywords:} Nuclear theory, Mean-field models, Nuclear reactions\\
  % Please give some freely chosen keywords that we can use in a
  % cumulative keyword index.
{\em Classification:}  17.16 Theoretical Methods - General, 17.22 
Hartree-Fock Calculations, 17.23 Fission and Fusion Processes\\
  %Classify using CPC Program Library Subject Index, see (
  % http://cpc.cs.qub.ac.uk/subjectIndex/SUBJECT_index.html)
  %e.g. 4.4 Feynman diagrams, 5 Computer Algebra.
{\em External routines/libraries:}  LAPACK, FFTW3. 
\\
{\em Catalogue identifier of previous version:}             \\
 AESW\_v1\_0.\\
  %Only required for a New Version summary, otherwise leave out.
{\em Journal reference of previous version:}                  \\
J.~A.~Maruhn, P.-G.~Reinhard, P.~D.~Stevenson, and A.~S.~Umar, "The
TDHF Code Sky3D", Comp. Phys. Comm. {\bf 185}, 2195 (2014).\\
  %Only required for a New Version summary, otherwise leave out.
{\em Does the new version supersede the previous version?:}   \\
Yes.\\

{\em Nature of problem:} The time-dependent Hartree-Fock equations can
be used to simulate nuclear vibrations and collisions between nuclei
for low energies. This code implements the equations based on a Skyrme
energy functional and also allows the determination of the
ground-state structure of nuclei through the static version of the
equations. For the case of vibrations the principal aim is to
calculate the excitation spectra by Fourier-analyzing the time
dependence of suitable observables. In collisions, the formation of a
neck between nuclei, the dissipation of energy from collective motion,
processes like charge transfer and the approach to fusion are of
principal interest.\\
  %Describe the nature of the problem here.
\\
{\em Solution method:} The nucleonic wave function spinors are
represented on a three-dimensional Cartesian mesh with no further
symmetry restrictions. The boundary conditions are always periodic for
the wave functions, while the Coulomb potential can also be calculated
for an isolated charge distribution. All spatial derivatives are
evaluated using the finite Fourier transform method. The code solves
the static Hartree-Fock equations with a damped gradient iteration
method and the time-dependent Hartree-Fock equations with an expansion
of the time-development operator. Any number of initial nuclei can be
placed into the mesh in with arbitrary positions and
initial velocities.\\

{\em Reasons for the new version:}\\
A few bugs were fixed and a number of enhancements added concerning
faster convergence, better stability, and more sophisticated analysis
of some results.

{\em Summary of revisions:}\\
The following is a brief summary. A more complete documentation can be
found as {\tt update.pdf} in the {\tt Documentation} subdirectory.

{\bf New documentation:} It was decided to switch the documentation to using the {\tt
    Doxygen} system (available from {\tt www.doxygen.org}), which can
  generate the documentation in a variety of formats. To generate the
  documentation, go into the {\tt Doc-doxygen} subdirectory and
  execute {\tt make html}, {\tt make latex}, or {\tt make all} to
  produce the corresponding version or both of them.

%{\color{red}
  The documentation inserted into the source files accounts for most
  of the formal changes in them. The general documentation is also
  updated and present as ``Documentation.pdf''.
%}

 {\bf Bug fixes:} 
\begin{enumerate}
\item In the force database {\tt forces.data} two digits were
  interchanged in the definition of SLy4d, leading to wrong results
  for that force.

\item If a restart is done for a two-body collision, the code changed
  the number of fragments to {\tt nof}=1. The restart is then
  initialized like a single-nucleus case with  {\tt nof}=1. But two-body
  analysis was activated only for {\tt nof}>1 such that it was absent
  after restart.

%{\color{red}
\item In the time-dependent mode, the wave functions were only save at
  intervals of {\tt mprint} and {\tt mrest}, respectively. If a
  calculation stops because of reaching the final distance or
  fulfilling the convergence criterion, this may lead to a loss of
  information, so that now both are done also in this event before the
  job finishes.
%}

\item The external field parameters were calculated directly from the
  input in {\tt getin\_external}. Since this is called before the
  fragment initialization is done, coefficients depending on proton or
  neutron number will not be calculated correctly. For this reason,
  the calculation of these coefficients is separated into a new
  routine {\tt init\_external}, which is called directly before the
  dynamic calculation starts.
\end{enumerate}

{\bf Array allocation:} It turned out that having the working arrays
as automatic variables could cause problems, as they are allocated on
the stack and the proper stack size must be calculated. Therefore in
all cases where a larger array is concerned, it is now changed to {\tt
  ALLOCATABLE} and allocated and deallocated as necessary.

{\bf Elimination of ``guru'' mode of FFTW3} While the guru mode as
defined in the FFTW3 package (see {\tt fftw.org}) offers an elegant
formulation of complicated multidimensional transforms, it is not
implemented in some support libraries like the Intel\textregistered\
MKL. There is not much loss in speed when this is replaced by
standard transforms with some explicit loops added where
necessary. This affects the wave function transforms in the $y$ and
$z$ direction.

{\bf Enhancement of the makefile} In the previous version there were
several versions of the makefile, which had to be edited by hand to
use different compilers. This was reformulated using a more flexible
file with various targets predefined. Thus to generate the executable
code, it is sufficient to execute ``{\tt make} {\em target} `` in the
{\tt Code} subdirectory, where {\em target} is one of the following:

\begin{itemize}
\item {\tt seq} : simple sequential version with the gfortran compiler.
\item {\tt ifort}, {\tt ifort\_seq} : sequential version using the Intel compiler. 
\item {\tt omp} and {\tt ifort\_omp} produce the OpenMP version for
  the gfortran and Intel compiler, respectively.
\item {\tt mpi} : MPI version, which uses the compiler mpif90.
\item {\tt mpi-omp} : MPI version also using OpenMP on each node.
\item {\tt debug}, {\tt seq\_debug}, {\tt omp\_debug}, {\tt
    mpi\_debug} : enable debugging mode for these cases. The first
  three use the gfortran compiler.
\item {\tt clean} : removes the generated object and module files.
\item {\tt clean-exec} : same as {\tt clean} but removes the
  executable files as well.
\end{itemize}

The generated executable files are called {\tt sky3d.seq}, {\tt
  sky3d.ifort.seq}, {\tt sky3d.mpi}, {\tt sky3d.omp}, {\tt
  sky3d.ifort.omp}, and {\tt sky3d.mpi-omp}, which should be
self-explanatory. Thus several versions may be kept in the code
directory, but a {\tt make clean} should be done before producing a
new version to make sure the object and module files are correct.

{\bf Skyrme-force compatibility for static restart:} the code normally
checks that the Skyrme forces for all the input wave functions
agree. It may be useful, however, to initialize a static calculation
from results for a different Skyrme force. Therefore the consistency
check was eliminated for the static case.

{\bf Acceleration of the static calculations:} The basic parameters
for the static iterations are (see Eq.~12 of the original paper) $x_0$
(variable {\tt x0dmp}), which determines the size of the gradient
step, and $E_0$ (variable {\tt e0dmp}) for the energy damping. These
were read in and never changed throughout the calculation, except
possibly through a restart. This can cause slow convergence, so that a
method was developed to change {\tt x0dmp} during the iterations. The
value from the input is now regarded as the minimum allowed one and
saved in {\tt x0dmpmin}. At the start of the iterations, however, {\tt
  x0dmp} is multiplied by 3 to attempt a faster convergence.

The change in {\tt x0dmp} is then implemented by comparing the HF
energy {\tt ehf} and the fluctuations {\tt efluct1} and {\tt efluct2}
to the previous values saved in the variables {\tt ehfprev}, {\tt
  efluct1prev}, and {\tt efluct2prev}. If the energy decreases or one
of the fluctuations decreases by a factor of less than $1-10^{-5}$,
{\tt x0dmp} is increased by a factor 1.005 to further speed up
convergence. If none of these conditions holds, it is assumed that the
step was too large and {\tt x0dmp} is reduced by a factor 0.8, but is
never allowed to fall below {\tt x0dmpmin}.

This whole process is turned on only if the input variable {\tt
  tvaryx\_0} in the namelist "static" is {\tt .TRUE.} The default
value is {\tt .FALSE.}

A speedup up to a factor of 3 has been observed.

{\bf External field expectation value} This value, which is printed in
the file {\tt external.res}, was calculated from the spatial field
including the (time-independent) amplitude {\tt amplq0}. The temporal
Fourier transform then becomes quadratic in the amplitude, as the
fluctuations in the density also grow linearly in {\tt amplq0}
(provided the perturbation is not strong enough to take it into the
nonlinear regime).  This may be confusing and we therefore divided the
expectation value by this factor.  Note that if the external field is
composed of a mixture of different multipoles (not coded presently),
an overall scale factor should instead be used.

%{\color{red}
{\bf Enhanced two-body analysis:} the analysis of the final two-body
quantities after breakup included directly in the code was very
simplified and actually it was superfluous to do this so frequently.
This is replaced by a much more thorough analyis, including
determination of the internal angular momenta of the fragments and of
a quite accurate Coulomb energy. It is done only when the final
separation is reached, while a simple determination of whether the
fragments have separated and, if so, what their distance is, is
performed every time step.
%}

 {\bf Diagonalization} In the original program the diagonalization of
 the Hamiltonian in the subroutine {\tt diagstep} was carried out
 employing an eigenvalue decomposition using the LAPACK routine {\tt
   ZHBEVD} which is optimized for banded matrices. This routine is
 replaced in the update by the routine {\tt ZHEEVD} which is optimized
 for general hermitian matrices. This change should result in a
 moderate speed up for very large calculations. Furthermore the array
 {\tt unitary}, previously a {\tt nstmax}$\times${\tt nstmax} array
 has been reduced to a {\tt nlin}$\times${\tt nlin} array, where {\tt
   nlin} is the number of wave functions of either neutrons or
 protons. This array is now used as input and output for {\tt ZHEEVD}.

{\bf New formulation of the spin-orbit term:}
The action of the spin-orbit term has been corrected to comply with a
strictly variational form. Starting from the spin-orbit energy
\begin{equation}
  E_{ls}
  =
  t_{ls}\int d^3r\,\vec{J}\cdot\nabla\rho
  \quad,
\end{equation}
we obtain by variation with respect to the s.p. wavefunction $\psi^*$
the spin-orbit term in the mean field in the symmetrized form
\begin{equation}
  \hat{h}_{ls}\psi
  =
  \frac{\mathrm{i}}{2}\left(
   \vec{W}\cdot(\vec{\sigma}\times\nabla)\psi
   +
   \vec{\sigma}\cdot(\nabla\times(\vec{W}\psi)
 \right)
 \label{eq:hls-symm}
\end{equation}
where $\vec{W}=t_{ls}\nabla\rho$. In the previous version of the
code, this term was simplified by applying the product rule for the
$\nabla$ operator yielding
\begin{eqnarray}
  \frac{\mathrm{i}}{2}\left(
   \vec{W}\cdot(\vec{\sigma}\times\nabla)\psi
   +
   \vec{\sigma}\cdot(\nabla\times(\vec{W}\psi)
 \right)
 &=&
  \mathrm{i}\vec{W}\cdot(\vec{\sigma}\times\nabla)\psi
  \quad.
\end{eqnarray}
Closer inspection reveals that the product rule is not perfectly
fulfilled if the $\nabla$ operator is evaluated with finite Fourier
transformation as inevitably done in the grid representation of the
code. It turned out that this slight mismatch can accumulate to
instabilities in TDHF runs over long times. Thus the variationally
correct form (\ref{eq:hls-symm}) has been implemented now, although it
leads to slightly longer running times.\\ 
\\
{\em Restrictions:} The reliability of the mean-field approximation is
limited by the absence of hard nucleon-nucleon collisions. This limits
the scope of applications to collision energies about a few MeV per
nucleon above the Coulomb barrier and to relatively short interaction
times. Similarly, some of the missing time-odd terms in the implementation
of the Skyrme interaction may restrict the applications to even-even
nuclei.\\
\\
{\em Unusual features:}\\
The possibility of periodic boundary conditions and the highly
flexible initialization make the code also suitable for astrophysical
nuclear-matter applications.\\
   \\
{\em Running time:} The running time depends strongly on the size of
the grid, the number of nucleons, and the duration of the
collision. For a single-processor PC-type computer it can vary betweeneen
a few minutes and weeks.\\
  %Give an indication of the typical running time here.
   \\
\end{small}
\bibliographystyle{elsarticle-num} 
\bibliography{update}
\end{document}
