
% skyax.tex: complements the Fortran code 'skyax.f'  
%
%
\documentstyle[12pt]{article}

\textwidth 16cm
\textheight 20.5cm
\topmargin  -1.5cm
\oddsidemargin -0cm
\evensidemargin -0cm
\parindent 0cm
\parskip 33pt
%\pagestyle{empty}



\newcommand{\st}{^+}
\newcommand{\ue}{^{\mbox{}}}
\newcommand{\dfrac}[2]{{\displaystyle\frac{#1}{#2}}}
\newcommand{\sfrac}[2]{{\textstyle\frac{#1}{#2}}}
\newcommand{\home}{\hbar\omega}
\newcommand{\be}{\begin{equation}}
\newcommand{\ee}{\end{equation}}
\newcommand{\eel}[1]{\label{#1} \end{equation} \marginpar{#1} }
\newcommand{\bea}{\begin{eqnarray}}
\newcommand{\eea}{\end{eqnarray}}
\newcommand{\cJ}{{\cal J}}
\newcommand{\ncJ}{\nabla {\cal J}}
\newcommand{\sna}{\sigma \cdot \nabla}
\newcommand{\psa}{\psi_\alpha}
\newcommand{\psap}{\psi_\alpha^{(+)}}
\newcommand{\psam}{\psi_\alpha^{(-)}}
\newcommand{\psapm}{\psi_\alpha^{(\pm)}}
\newcommand{\mmp}{m_\alpha^{(+)}}
\newcommand{\mmm}{m_\alpha^{(-)}}
\newcommand{\up}[1]{^{\mbox{\small(#1)}}}
\newcommand{\comm}[1]{\\{\sl Comment: #1}}


\setlength{\unitlength}{1cm}




\begin{document}

%\renewcommand{\baselinestretch}{1.8}
\renewcommand{\topfraction}{0.8}
\renewcommand{\textfraction}{0.2}
\renewcommand{\floatpagefraction}{0.8}
\renewcommand{\arraystretch}{1.5}



\title{Iteration scheme with constraints}
\author{P.-G. Reinhard}
\date{7. November 2007}
\maketitle

Here, we summarize the constraint iteration scheme published first in
{``Density as a constraint and the separation of internal excitation
  energy in {TDHF}''}, {R.Y. Cusson et al}, {Z. Phys. A }, {\bf 320}
(1985) 475 and which has been succesfully been used in several
applications since. (Note that there are similarities to the ``augmented
Lagrangian'' method which have not yet been fully worked out.)


The constrained problem is:
\be
    E = \mbox{stationary}
    \qquad , \qquad
    \langle Q \rangle = Q_0
\ee
Variation yields the EoM
\be
    \frac{\partial E}{\partial \varphi^*} = h \varphi = \varepsilon \varphi
    \qquad \mbox{together with} \qquad
    h \perp Q
\ee
This is solved by adding a constraining force to the functional, i.e.
\be
    \frac{\partial (E-\lambda \langle Q \rangle)}{\partial \varphi^*}
    =
    (h-\lambda) \varphi = \varepsilon \varphi
\ee
Thus far the theoretical foundations.

The practical iteration goes as follows:
\begin{description}
 \item{{\bf Start:}} Given a stage with
       \begin{center}
       \begin{tabular}{ll}
       wavefunction & $\Phi^{(n)}$ \\
       Lagrange parameter & $\lambda^{(n)}$ \\
       actual Q-momentum &
          $Q^{(n)} = \langle\Phi^{(n)}|Q|\Phi^{(n)}\rangle$
       \end{tabular}
       \end{center}
 \item{{\bf $h$-step:}} Do one normal wavefunction iteration
       \be
        |\tilde{\Phi}\rangle
        =
        {\cal O} \left(1-{\cal D}[h-\lambda^{(n)}Q]\right)
           |\Phi^{(n)}\rangle
       \label{hstep}
       \ee
       where ${\cal O}$ means orthonormalization and ${\cal D}$
       damping.
       This yields a new value for the deformation
       $$
         \tilde{Q} = \langle\tilde{\Phi}|Q|\tilde{\Phi}\rangle
       $$
\item{$Q$-correction:} It is very probable that $\tilde{Q}\neq Q_0$.
       Correct for this by
       \be
        |\Phi^{(n+1)}\rangle
        =
        {\cal O} \left(1-\gamma Q\right) |\tilde{\Phi}\rangle
       \label{Qstep}
       \ee
       The stepsize is
       \be
         \gamma = \frac{\tilde{Q}-Q_0}{2\langle Q_{ph}^2\rangle}
       \label{Qsize}
       \ee
       which is adjusted such that
       $$
         Q^{(n+1)} \approx \tilde{Q} - 2\gamma \langle Q_{ph}^2\rangle
         = Q_0
       $$
\item{{\bf Update $\lambda$:}} The fact that
       $$
         \tilde{Q} \neq Q^{(n)}
       $$
       hints that the $\lambda^{(n)}$ did not yet correctly project
       out the $Q$-component in $h$. We correct
       \be
         \lambda^{(n+1)} = \lambda^{(n)} + \delta \lambda
         \qquad , \qquad
          \delta \lambda =
          \frac{\langle Q_{ph} \left(h - \lambda^{(n)} Q\right)\rangle}
               {\langle Q_{ph}^2\rangle}
       \label{lstep}
       \ee
       This update aims at fulfilling
       $$
        \langle Q_{ph} \left(h - (\lambda^{(n)} + \delta \lambda)Q\right)
        \rangle = 0
       $$
\end{description}
Thus far the "ideal" prescription. Reality has always its own problems.
First, we have linearized throughout and this is not justified in the
early stages of the iteration. Second, the information on
$\langle Q_{ph}^2\rangle$ is rather expensive in large spaces.
And third, the $\langle Q_{ph} \left(h - \lambda^{(n)} Q\right)\rangle$
adds another even more expensive bit. It does not pay to perform these
expensive expressions if we have an uncertainty to start with. The
uncertainties may very well drive the iteration unstable and we have
to introduce stabilizing measres anyway. Those measures are usually
to complement the steps by some numerical parameters which slow the
step done but, at the same time, stabilize him. Once we have those
parameters in the scheme, we may very well also approximate the costly
matrix elements. Thus we use instead of eq.~(\ref{Qsize}) for
the size of the corrective step the approximation
\be
  \gamma = c_0 \frac{\tilde{Q}-Q_0}{2\langle Q^2\rangle}
\ee
where $c_0$ is an extra parameter to stabilize the
correction and where we have taken the simpler straightforward
$\langle Q^2\rangle = \int dr \rho(r) Q(r)^2$ for the variance of $Q$.
Note that this reduces the step the more the larger $\langle Q \rangle$
is. The step for $\lambda$ can be approximated by using the
difference $\tilde{Q}-Q^{(n)}$. This yields
\be
    \delta \lambda = \epsilon_q \frac{E_0}{x_0}
    \frac{\tilde{Q}-Q^{(n)}}{2\langle Q_{ph}^2\rangle}
\ee
where, again a numerical parameter, $\epsilon_q$, has been added
for extra stabilization of the update formula.








\end{document}
